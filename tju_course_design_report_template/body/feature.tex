\chapter{项目特色}
\section{数据库连接池}
\subsection{背景}

\subsubsection{原始JDBC连接缺陷}
在项目一和项目三中,连接数据库的方式是传统的JDBC连接。每进行一次SQL操作就需要经过创建连接,操作,释放连接的步骤,所有连接都是临时短暂的。

\subsubsection{出现问题}
Mysql 查询大量数据异常:

默认情况下,Windows允许用于使用5000个临时TCP端口。任何端口关闭后,它将在TIME WAIT状态保持120秒。与重新初始化全新的连接相比,该状态允许以更低的开销重新使用连接。但是,在该时间逝去前,无法再次使用该端口。

如果TCP端口堆栈较少,以及具有TIME WAIT状态的大量在短时间内打开和关闭的 TCP端口,就很可能遇到端口耗尽问题。

\subsection{优化方法}
连接池技术:一种池化技术。在程序第一次启动的时候,程序会初始化连接池,连接池里面存放的是连接Mysql数据库的连接,并且统一管理释放。当有线程需要进行SQL操作时,他会从线程池里取一个连接,
操作完成之后将连接归还回连接池。这样就将短命临时的数据库连接变成了持久性的连接,有效地解决了临时端口消耗过快,后端与数据库连接消耗过多时间的问题。

\subsection{使用效果}

压力测试:查看连续进行10000次SQL搜索的时间

采用连接池前:
\begin{figure}[htbp]
	\centering
	\includegraphics[width=0.8\textwidth]{lianjiechia}
	\caption{优化后}
	\vspace{\baselineskip}
\end{figure}

 
采用连接池后:
\begin{figure}[htbp]
	\centering
	\includegraphics[width=0.8\textwidth]{lianjiechia}
	\caption{优化后}
	\vspace{\baselineskip}
\end{figure}
\section{数据库缓存}
\subsection{背景}

当网站的处理和访问量非常大的时候,数据库的压力就变大了,数据库的连接池,数据库同时处理数据的能力就会受到很大的挑战,一旦数据库承受了其最大承受能力,网站的数据处理效率就会大打折扣。

\subsection{技术原理}
Redis其实就是说把表中经常访问的记录放在了Redis中,然后用户查询时先去查询Redis再去查询MySQL,确实实现了读写分离,也就是Redis只做读操作。由于缓存在内存中,所以查询会很快。

同时还需注意数据库的同步。在对Mysql数据库进行增删改的操作时,会删除redis数据库中相关联的信息,在下次查询时将数据填入redis缓存,做到缓存和数据库的同步。

\subsection{redis优势}
redis缓存在内存中,同时是类似Hashmap的存储方式,查询速度很快。

\subsubsection{使用效果}

压力测试:通过JMeter做压力测试,通过多线程并发不断地给后台发送HTTP请求,测试性能

采用redis前:
\begin{figure}[htbp]
	\centering
	\includegraphics[width=0.8\textwidth]{redisa}
	\caption{优化前}
	\vspace{\baselineskip}
\end{figure}


采用redis后:
\begin{figure}[htbp]
	\centering
	\includegraphics[width=0.8\textwidth]{redisb}
	\caption{优化后}
	\vspace{\baselineskip}
\end{figure}

可以看到吞吐量和延迟都有了很明显的优化效果
\section{积分系统}

\section{修改用户信息}


\section{使用vue3}
此次项目中,我们放弃了原来的vue2,采用vue3进行前端开发,性能得到了显著的提升,具体优点如下:
\begin{enumerate}
	\item {diff算法的优化}:vue3新增了静态标记(patchflag),在Vue2中,每次更新diff,都是全量对比,Vue3则只对比带有标记的,这样大大减少了非动态内容的对比消耗.
	\item {hoistStatic静态提升}:vue2无论元素是否参与更新,每次都会重新创建然后再渲染。vue3对于不参与更新的元素,会做静态提升,只会被创建一次,在渲染时直接复用即可。
	\item {cacheHandlers事件侦听器缓存}:vue2中,绑定事件每次触发都要重新生成全新的function去更新;Vue3中,cacheHandlers是事件缓存对象,当cacheHandlers开启,会自动生成一个内联函数,同时生成一个静态节点。当事件再次触发时,只需从缓存中调用即可,无需再次更新。
	\item {ssr渲染}:渲染时,若存在大量静态内容,这些内容会被当作纯字符串推进一个buffer里面,即使存在动态的绑定,也会通过模版插值潜入进去。这样会比通过虚拟dmo来渲染的快上很多。
	\item {按需编译,体积比vue2更小}:在 Vue 3 中,通过将大多数全局API和内部帮助程序移至ES模块导出来,减小了框架。这使现代的打包工具可以静态分析模块依赖性并删除未使用的导出相关的代码。模板编译器还会生成友好的 Tree-shaking 代码,在模板中实际使用了该功能时才导入该功能的帮助程序。
	\item {支持多根节点组件}:在vue2中,不支持多根组件,当用户意外创建多根组件时会发出警告,而vue3中,组件可以有多个根节点。
\end{enumerate}